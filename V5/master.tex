\def\documentType{Protocol}
\def\mainLanguage{ngerman}

\input{../template.tex}

\pgfqkeys{/TemplateVersion0}{
    properties/Authors 				= {Tom Mohr&&Martin Ohmeyer},
    properties/Title	 			= {Versuch V5},
    properties/Subtitle 			= {C405 Hardwarepraktikum II},
    properties/Institution 			= {HTWK Leipzig},
    properties/Group 				= {23INB-3},
    properties/Versioning 	      	= {true},
    misc/Date/Prefix				= {Stand: },
    misc/Page/CountPrefix			= {Seiten: },
    misc/Group/Prefix				= {Gruppe: },
    Protocol/Comment		 	    = {Abnahme: 20. Januar 2025}
}

\usepackage{tikz}
\usepackage[table]{xcolor}
\usepackage[utf8]{inputenc}
\usepackage{color,soul}
\usepackage{mathtools}
\usepackage{transparent}
\usepackage{adjustbox}
\usepackage{tcolorbox}
\usepackage{inconsolata}
\usetikzlibrary{decorations.pathreplacing}
\usetikzlibrary{arrows, automata, positioning}
\graphicspath{{../images/}}

\begin{document}

	\chapter{Schaltplan}
        \begin{figure}[H]
            \centering
            \includegraphics[width=15cm]{schaltplan}
            \caption{Schaltplan}
            \label{fig:schaltplan}
        \end{figure}

	\chapter{PCB Design}
        \begin{figure}[H]
            \centering
            \includegraphics[width=15cm]{pcb}
            \caption{Entwurfsansicht Leiterplatte}
            \label{fig:pcb-entwurf}
        \end{figure}

	\chapter{Produkt}
		\begin{figure}[H]
			\centering
			\includegraphics[height=0.8\textheight]{platine_top}
			\caption{Platine (Vorderseite)}
			\label{fig:product-top}
		\end{figure}

		\begin{figure}[H]
			\centering
			\includegraphics[height=0.8\textheight]{platine_bottom}
			\caption{Platine (Rückseite)}
			\label{fig:product-bottom}
		\end{figure}

\end{document}