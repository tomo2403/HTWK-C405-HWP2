\chapter{Kodierung}
\section{Notwendigkeit}
\label{sec:notwendigkeit}
Nacheinander gesendete Nibble müssen auf der Leitung voneinander unterscheidbar sein. Um dies zu gewährleisten, wird ein Taktsignal in den Datenstrom kodiert, indem verhindert wird, dass ein gesendetes Nibble gleich seinem Vorgänger ist. Dies verursacht den Sonderfall zweier gleicher aufeinanderfolgender Nibble. Zu dessen Behandlung muss eine Escape-Sequenz definiert werden, welche die gleichen Nibble durchbricht. Durch die Einführung einer solchen Escape-Sequenz wird ihr eigenes Auftreten im Datenstrom jedoch selbst zu einem Sonderfall. Zur Handhabung dieser Sonderfälle und zur Definition von Blöcken wurde ein Protokoll zwischen Sender und Empfänger vereinbart.

\section{Besondere Sequenzen}
Aus den in \ref{sec:notwendigkeit} dargelegten Gründen werden vordefinierte Sequenzen in den Datenstrom injiziert. Eine solche Sequenz setzt sich aus vier festen (Escape-Sequenz) und vier dynamischen Bits (Handlungsanweisung: "Command") zusammen.

\begin{figure}[H]
    \centering
    \[
        \underbrace{\text{XXXX}}_\text{\normalsize esc} \ \overbrace{\text{XXXX}}^\text{\normalsize cmd}
    \]
    \caption{Aufbau einer Kodierungssequenz}
\end{figure}

Die \textbf{Escape-Sequenz} trennt kodierende Sequenzen vom restlichen Datenstrom ab. Sie selbst hält keine Information darüber, um welche Kodierung es sich handelt. Aufgrund ihrer Sonderfunktion darf sie nicht regulär im Datenstrom auftreten und muss ggf. selbst escaped werden.

\textbf{Commands} erhalten erst dann ihre Bedeutung, wenn sie unmittelbar nach der Escape-Sequenz stehen. Sie geben Auskunft darüber, um welche Kodierung es sich handelt und implizieren, wie sich ein Dekodierender verhalten muss, um die originalen Daten wieder zu rekonstruieren. Ist das nachfolgende Nibble auf einen Command (im binären) mit diesem identisch, wird anstelle des normalen Commands dessen Fallback-Version genutzt.

\begin{table}[H]
    \center
    \def\arraystretch{1.3}
    \rowcolors{2}{gray!15}{white}
    \begin{tabular}{|c|l|l|l|}
        \rowcolor{gray!50}
        \hline
        \textbf{Hex} & \textbf{Bezeichnung}          & \textbf{Bedeutung}                 \\
        \hline
        0            & escapeSequence                & Das nächste Nibble ist ein Command \\
        1            & beginDataBlockDefault         & Ein Datenblock beginnt             \\
        2            & beginDataBlockFallback        & Ein Datenblock beginnt             \\
        3            & beginControlBlockDefault      & Ein Kontrollblock beginnt          \\
        4            & beginControlBlockFallback     & Ein Kontrollblock beginnt          \\
        5            & endBlockDefault               & Der aktuelle Block endet           \\
        a            & endBlockFallback              & Der aktuelle Block endet           \\
        6            & insertPrevNibbleAgainDefault  & Ein doppeltes Nibble im Datenstrom \\
        7            & insertPrevNibbleAgainFallback & Ein doppeltes Nibble im Datenstrom \\
        8            & insertEscSeqAsDataDefault     & Die Esc-Seq trat im Datenstrom auf \\
        9            & insertEscSeqAsDataFallback    & Die Esc-Seq trat im Datenstrom auf \\
        \hline
    \end{tabular}
    \caption{Bitsequenzen und ihre Bedeutung}
    \label{tab:escape_sequences}
\end{table}

\section{Datenblöcke}